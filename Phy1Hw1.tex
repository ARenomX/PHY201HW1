\documentclass{article}

\usepackage{amsmath}
\usepackage{amssymb}
\usepackage{textcmds}
\usepackage{graphicx}
\usepackage{subcaption}
\usepackage{float}
\usepackage{hhline}
\usepackage[parfill]{parskip}

\title{PHY201: Homework 1}

\date{7$^{th}$ November 2019}
\author{Ariel Attias\\Matthieu Chapuy\\Matthieu Melennec\\Andr\'e Renom}

\begin{document}


	\pagenumbering{gobble}
	\maketitle
	\tableofcontents
	\newpage
	\pagenumbering{arabic}

\section{Horizontally Excited Pendulum}

\subsection{} % Exercise 1.1

Under normal assumptions, we would write the position of the pendulum as:
\begin{align*}
	\left( \begin{matrix} x \\ z \end{matrix} \right) = \left( \begin{matrix}x_f +  l\, sin\theta \\ -l\, cos\theta \end{matrix} \right)
\end{align*}
However for $\theta \ll 1$, we can use the Taylor series of the trigonometric functions to approximate $cos\theta \approx 1$ and $sin\theta \approx \theta$. We therefore have:
\begin{align*}
	\left( \begin{matrix} x \\ z \end{matrix} \right) = \left( \begin{matrix}x_f +  l\theta \\ -l \end{matrix} \right)
\end{align*}

\subsection{} % Exercise 1.2

We now consider the forces in the direction of the bar at the point mass. We observe two forces: the constraint force in the bar, that acts parallel to the bar, and the force due to gravity, that acts straight down. By projecting the force due to gravity along the direction of the bar, we can write the equation
\begin{align*}
	F_{bar} - mg\,cos\theta = ma
\end{align*}
with $a$ the component of the acceleration in the direction of the bar. However we know the bar to be rigid, hence $a = 0$, and as established before, over the studied range, $cos\theta = 1$. We can therefore re-write the equaiton above as
\begin{align*}
	F_{bar} = mg
\end{align*}

\subsection{} % Exercise 1.3

In order to write the oscillator equation of the system, we will consider the motion of the mass in the direction of the $x$-axis. From our definition of the $x$ position of the particle in \textbf{1.1}, we can derive twice to obtain the acceleration of the particle in the $x$-direction, $\ddot{x} = \ddot{x}_f + l \ddot{\theta}$.\\

\noindent By considering the forces on the mass, we get that:
\begin{align*}
	-F_{bar}sin\theta - \eta \dot{x} = ma\\
	-mg\,sin\theta - \eta \dot{x} = m( \ddot{x}_f + l \ddot{\theta})\\
	-g\,sin\theta - \frac{\eta}{m} \dot{x} = -\omega^2x_0sin \omega t + l \ddot{\theta}
\end{align*}
From this we can write our oscillator equation:
\begin{align*}
	 \frac{d^2 \theta}{dt^2} + \frac{\eta}{m}\frac{d \theta}{dt} + \frac{g}{l}\theta  = -\frac{\omega^2}{l}x_0sin\omega t 
\end{align*}

\subsection{} % Exercise 1.4

Replacing the expression of $x_f(t)$ by it's complex equivalent, we get the equation:
\begin{align*}
	\frac{d^2 \theta}{dt^2} + \frac{\eta}{m}\frac{d \theta}{dt} + \frac{g}{l}\theta  = -i\frac{\omega^2}{l}x_0e^{i \omega t} 
\end{align*}
We then consider a solution of  type $\theta = \Theta e^{i \omega t}$.\\
By inserting this into the equation above, we get:
\begin{align*}
	\omega^2\Theta e^{i \omega t} - i\frac{\eta \omega}{m}\Theta e^{i \omega t} - \frac{g}{l}\Theta e^{i \omega t} = i\frac{\omega^2}{l}x_0e^{i \omega t}
\end{align*}
We therefore get
\begin{align*}
	\Theta &= \frac{i \frac{\omega^2}{l}x_0}{-\omega^2 - i\frac{\eta \omega}{m} - \frac{g}{l}}\\
	&=
\end{align*}

\subsection{} % Exercise 1.5



\subsection{} % Exercise 1.6

\section{Equilibrium of Two Masses}

\subsection{} % Exercise 2.1

In order to use $z_1$ as a generalised coordinate for the system, we would have to be able to derive$u_2$ from it. However, since the two masses are connected by an inextensible filament of length $l$ and placed on a rigid surface, this is in fact the case. Specifically:
\begin{align*}
	u_2 = l - z_1
\end{align*}

\subsection{} % Exercise 2.2

\begin{align*}
	\mathcal{L} &= E_{K} - E_P\\
	\mathcal{L}(z,\dot{z},t) &= \frac{1}{2}(m_1+m_2)\dot{z}^2 - g(m_1z + m_2(l-z)sin\,\alpha)\\
	&= \frac{1}{2}(m_1+m_2)\dot{z}^2 - (m_1 - m_2sin\,\alpha)g z - m_2 l sin(\alpha)
\end{align*}

\subsection{} % Exercise 2.3

We consider the Euler-Lagrange equation:
\begin{align*}
	\frac{\partial \mathcal{L}}{\partial z} &= \frac{d}{dt}\frac{\partial \mathcal{L}}{\partial \dot{z}}\\
	 (m_1 - m_2sin\,\alpha)g &= \frac{d}{dt}(m_1 + m_2)\dot{z}\\
	 (m_1 - m_2sin\,\alpha)g &= (m_1 + m_2)\ddot{z}
\end{align*}

\subsection{} % Exercise 2.4

From the Euler-Lagrange equation gives us that:
\begin{align*}
	\ddot{z} = \frac{ m_1g - m_2gsin\,\alpha}{m_1 + m_2}
\end{align*}
In order to have equilibrium, we need $\ddot{z} = 0$. This can only be obtianed if $m_2sin\,\alpha = m_1$. Since $sin\,\alpha \leq 1$, This is possible if $m_2 > m_1$ but impossible if $m_2 < m_1$.

\section{Suspended Bar}

\subsection{One Mass Only}

\subsubsection{} % Exercise 3.1.1



\subsubsection{} % Exercise 3.1.2



\subsubsection{} % Exercise 3.1.3



\subsubsection{} % Exercise 3.1.4



\subsubsection{} % Exercise 3.1.5



\subsubsection{} % Exercise 3.1.6



\subsubsection{} % Exercise 3.1.7


\subsection{Connected Masses}

\subsubsection{} % Exercise 3.2.1



\subsubsection{} % Exercise 3.2.2



\subsubsection{} % Exercise 3.2.3



\subsubsection{} % Exercise 3.2.4



\end{document}