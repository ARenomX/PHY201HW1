\documentclass{article}

\usepackage{amsmath}
\usepackage{amssymb}
\usepackage{textcmds}
\usepackage{graphicx}
\usepackage{subcaption}
\usepackage{float}
\usepackage{hhline}
\usepackage[parfill]{parskip}

\title{PHY201: Homework 1}

\date{7$^{th}$ November 2019}
\author{Ariel Attias\\Matthieu Chapuy\\Matthieu Melennec\\Andr\'e Renom}

\begin{document}


	\pagenumbering{gobble}
	\maketitle
	\tableofcontents
	\newpage
	\pagenumbering{arabic}

\section{Horizontally Excited Pendulum}

\subsection{} % Exercise 1.1

Under normal assumptions, we would write the position of the pendulum as:
\begin{align*}
	\left( \begin{matrix} x \\ z \end{matrix} \right) = \left( \begin{matrix}x_f +  l\, sin\theta \\ -l\, cos\theta \end{matrix} \right)
\end{align*}
However for $\theta \ll 1$, we can use the taylor series of the trigonometric functions to approximate $cos\theta \approx 1$ and $sin\theta \approx \theta$. We therefore have:
\begin{align*}
	\left( \begin{matrix} x \\ z \end{matrix} \right) = \left( \begin{matrix}x_f +  l\theta \\ -l \end{matrix} \right)
\end{align*}
\subsection{} % Exercise 1.2



\subsection{} % Exercise 1.3



\subsection{} % Exercise 1.4



\subsection{} % Exercise 1.5



\subsection{} % Exercise 1.6

\section{Equilibrium of Two Masses}

\subsection{} % Exercise 2.1



\subsection{} % Exercise 2.2



\subsection{} % Exercise 2.3



\subsection{} % Exercise 2.4




\section{Suspended Bar}

\subsection{One Mass Only}

\subsubsection{} % Exercise 3.1.1



\subsubsection{} % Exercise 3.1.2



\subsubsection{} % Exercise 3.1.3



\subsubsection{} % Exercise 3.1.4



\subsubsection{} % Exercise 3.1.5



\subsubsection{} % Exercise 3.1.6



\subsubsection{} % Exercise 3.1.7


\subsection{Connected Masses}

\subsubsection{} % Exercise 3.2.1



\subsubsection{} % Exercise 3.2.2



\subsubsection{} % Exercise 3.2.3



\subsubsection{} % Exercise 3.2.4



\end{document}